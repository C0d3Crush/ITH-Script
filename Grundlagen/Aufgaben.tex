\section*{1}
\subsection*{1.1}
    Wofür steht die Notation \([n]\)
    \vspace*{0.5cm}
    \\
    A: Es gibt \(n \in \mathbb{N}_0\) mit \([n] = \{1, \cdots, n\}\) und \([n]_0 = \{0, 1, \cdots, n\}\) 
\subsection*{1.2}
    Bestimme die Mengen.
    \begin{itemize}
        \item [a)]\([3]_0 = \{0, 1, 2, 3\}\)
        \item [b)]\([0]_0 = \{0\}\)
        \item [c)]\([0] = \varnothing\)
    \end{itemize}

\section*{2}
\subsection*{2.1}
Wofür steht die Notation \(A^n\) mit einer Menge A?
\vspace*{0.5cm}
\\
A: Für eine Menge A und \(n \in \mathbb{N}\) ist \(A^n = \{(a_1, \cdots, a_n) : a_1, \cdots, a_n \in A\}\)
\subsection*{2.2}
\subsubsection*{2.2.1}
    \begin{enumerate}
        \item [] \(A = \{1, 2\}\)
        \item [] \(A^2 = \{(1,2), (2,1), (2,2),(1,1)\}\)
    \end{enumerate}

\subsubsection*{2.2.2}
    \begin{enumerate}
        \item [] \(M = \{"a", "bc", "d"\}\)
        \item [] \(M = \{(a, a, a), \cdots\}\)
    \end{enumerate}
    \(\rightarrow\) Möglichkeiten von \(M^3 : 3^3 = 27\)

\section*{3.}
Welche Mächtigkeit hat \(A^3\) mit \(A = [4]_0\)? 
\vspace*{0.5cm}
\\
\(A : |A|^3 = 5^3 = 125\)

\section*{4.}
Was ist eine n-äre partielle Funktion?
\vspace*{0.5cm}
\\
Für \(n \in \mathbb{N}\) ist \(\varphi\) eine n-äre partielle Funktion \(\varphi : A^n \leadsto B\)
mit 
\begin{itemize}
    \item \(dom(\varphi) \subseteq A^n\) \\Die Definitionsmenge ist teilmenge der n tupel aus A.
    \item \(im(\varphi) \subseteq B\) \\Der Wertebereich ist aus B.
\end{itemize}
\textit{ Der Pfeil \(\leadsto\) weist darauf hin, dass es sich um eine partielle Funktion handelt.}

\section*{5.}
Ist \(\log : (0, \infty) \to \mathbb{R}\) partiell berechenbar?
\vspace*{0.5cm}
\\
Obwohl der log auf \((0, \infty)\) nicht definiert ist, können wir schreiben das die funktion
\(\log : \mathbb{R} \leadsto \mathbb{R}\)
\begin{itemize}
    \item mit dem Wertebereich \(dom(\log) = (0,\infty) \subseteq \mathbb{R}\)
    \item und dem Bild \(im(\log) = \mathbb{R}\)
\end{itemize}
partiell berechenbar ist.

\section*{6.}
Was bedeuted Notation \(\varphi(x_1, \cdots, x_n)\uparrow\) bzw \(\downarrow\)
\vspace*{0.5cm}
\\
Für \(a_1, \cdots, a_n \in A\)
\begin{itemize}
    \item  bedeuted \(\varphi(a_1, \cdots, a_n) \downarrow\)\\ 
    dass \((a_1, \cdots, a_n) \subseteq dom(\varphi)\) gilt
    \item  und \(\varphi(a_1, \cdots, a_n \uparrow)\) bedeuted,\\
    dass \(a_1, \cdots, a_n \not \subseteq dom(\varphi)\)
\end{itemize}
\textit{Statt \(\varphi (a_1, \cdots, a_n) \uparrow\) schreiben wir auch \(\varphi(a_1, \cdots, a_n) = \uparrow\)}
\subsection*{7.}
    Gilt \(\uparrow\) oder \(\downarrow\)?
    \begin{itemize}
        \item \(\log(-5) \uparrow\)
        \item \(\log(7 \downarrow)\)
        \item \(f(x) = \frac{1}{x}\) und \(f: \mathbb{R} \leadsto \mathbb{R}\) so dass \(f(0) \uparrow\), aber \(f (x) \downarrow \) für alle \(x \in \mathbb{R} / \{0\}\)
    \end{itemize}
    \paragraph*{Anmerkung}
        Man kann auch \(\uparrow\) in Definitionen verwenden so dass \(g : \mathbb{Z}\leadsto \mathbb{Z}\)
        \[g(x) = 
        \begin{cases}
            1 &, \text{wenn x gerade}  \\
            \uparrow & \, \text{sonst}
        \end{cases}
        \]
    \subsection*{8.}
        Was ist \(dom(g)\) ?
        \vspace*{0.5cm}
        \\
        A: Die Menge der geraden ganzen Zahlen.

\subsection*{9.1}
    Was ist eine totale Funktion
    \vspace*{0.5cm}
    \\
    Die partielle Funktion \(\varphi\) ist \textbf{total}, wenn \(dom(\varphi) = A^n\) gilt.
\subsection*{9.2}
    \begin{itemize}
        \item [a)]\(exp : \mathbb{R} \to \mathbb{R}\) ist total.
        \item [b)]\(\log : \mathbb{R} \leadsto \mathbb{R}\) ist \textbf{nicht} total, da \(dom (\log) = (0,\infty) \not \subseteq \mathbb{R}\)
        \item [c)]\(\log : (0, \infty) \leadsto \mathbb{R}\) ist aber total. 
    \end{itemize}
    Totalität ist eine Eigenschaft die 'relativ' zum Definitionsbereich zu sehen ist.
\section*{10.}
    Was muss man verändern, um eine beliebige partielle Funktion \(\varphi: A^n \mapsto B\) zu einer totalen Funktion umzuschreiben?
    \vspace*{0.5cm}
    \\
    A: Man setze \(\hat{\varphi} : dom(\varphi) \to B\) mit \(\hat{ \varphi} (x) = \varphi(x)\) für alle \(x \in dom(\varphi)\) 
    
\section*{11} 
    \subsection*{11.1}
        Was ist eine Lineare (totale) Ordnung, auf einer Menge A?  
        \vspace*{0.5cm}
        \\
        \[\leq \subseteq A^2\]
        \begin{itemize}
            \item \(\leq\) ist der name der Teilmenge mit einem etwas komischen namen.
            \item \(\subseteq A^2\) deutet darauf hin das die Teilmenge aus Zweier-Tupeln besteht.
        \end{itemize}
        So dass folgende Eigenschaften erfüllt sind:
        \begin{enumerate}
            \item Reflexivität
            \item Antisymetrie
            \item Transitivität
            \item Totalität
        \end{enumerate}

    \subsection*{11.2}
        Was ist die Absicht hinter der Definition der linearen Ordnung?
        \vspace*{0.5cm}
        \\
        A: Eine Lineare Ordnung ist eine erweiterung der Begriffe on 'größer' und 'kleiner' wie man ie z.b. \(\mathbb{N}\) kennt. Das ziel ist es diese Begriffe in sinnvoller weise auf eine beliebige Menge zu erweitern.
    
    \subsection*{11.3}
        