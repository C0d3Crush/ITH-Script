\coversection{ReguläreSprachen/regul.png}{Komplexitätsklassen P und NP}
{}

\mysubsection{Definition(Komplexitätsklassen P, FP und NP)*}
    Wir setzen 
    \begin{itemize}
        \item \(\textbf{P} := \textbf{DTIME} (poly),\)
        \item \(\textbf{FP} := \textbf{FTIME}(poly),\)
        \item \(\textbf{NP} := \textbf{NTIME}(poly)\)
    \end{itemize}
    wobei poly := \{\(n^c + c, c \in \mathbb{N}_0\)\}
    Offenbar gilt \textbf{P} \(\subseteq\) \textbf{NP}, allerding ist unklar ob \textbf{P} \(\not \subseteq\) \textbf{NP}.

\mysubsection{Definition (p-m-Reduktion)}
    Eine Sprache A über \{0, 1\} ist in \textbf{polynomieller Zeit manny-to-one- reduzierbar}, auch \textbf{p-m-reduzierbar}, kurz \(A \leq^P_m B\), auf eine Sprache B über \{0, 1\}, wenn es eine Funktion \(f \in\textbf{FP}\) gibt, so dass 
    \[
        w\in A \Leftrightarrow f (w) \in B, \quad \forall w \in \{0, 1\}^*  
    \] 
    gilt. Gelte \(A \leq^P_m\) und \(B \leq^P_m\), so sind A und B \textbf{p-m-äquivalent}, kurz \(A=^P_m B\)

\mysubsection{Bemerkung(Eigenschaften der p-m-Reduktion)*}
    \begin{itemize}
        \item [(i)] \(\leq^P_m\) transitiv
        \item [(ii)] Seien A und B Sprachen mit \(A \leq ^P_m B\). Aus \(B \in \textbf{P}\), folgt \(A \in \textbf{P}\) und aus \(B \in \textbf{NP}\) folgt \(A \in \textbf{NP}\).
        \item [(iii)] Alle Sprachen \(L \in \textbf{P}\) mit \(\varnothing \not = L \not = \{0,1\}^*\) sind p-m-äquivalent.
    \end{itemize}

\mysubsection{Definition (\textbf{NP}-schwer, \textbf{NP}-vollständig)}
    Eine Sprache S wird als \textbf{NP}- Schwer bezeichnetm, wenn \(L \leq ^P_m S \quad \forall L \in \textbf{NP}\) gilt. Gilt zusätzlich \(S \in \textbf{NP}\), so wird S als \textbf{NP}-vollständig bezeichnet.

    \paragraph*{Nächstes ziel:} 
        Finden eines NP-vollständigen Problems.

\mysubsection{Definition (aussagenlogische Formel)}
    Sei Var eine abzählbare Menge mit \(\lnot, \wedge \not \in Var\). Die Menge der \textbf{aussagenlogischen Formeln} A ist die l. Inklusion kleinster Menge von Wörtern über \(Var \cup \{\lnot, \wedge, (,)\}\) mit \(a \in A \quad \forall a \in Var\) und \(\lnot \varphi, (\varphi \wedge \psi) \in A \quad \varphi, \psi \in A\). Für \(\varphi, \psi, \in A\) schreiben wir statt \(\lnot(\lnot \varphi \wedge \lnot \psi)\) auch \((\varphi \vee \psi)\) und statt (\(\lnot \varphi \vee \psi\)) auch (\(\varphi \to \psi\)). Wie üblich vereibaren wir Klammerregeln zur besseren lesbarkeit: \\ \(\lnot\) bidet stärker als \(\wedge\) und \(\vee\). 
    \paragraph*{Beispielsweise}
        \(\lnot a \wedge b \wedge c = ((\lnot a \wedge b)\wedge c)\)
    
\mysubsection{Definition(Grundbegriffe der Aussagenlogik)\(\diamondsuit\)}
    \begin{itemize}
        \item [(i)]DieElemente von Var heißen \textbf{Variablen}.
        \item [(ii)] Für \(\varphi, \varphi_1, \cdots, \varphi_n \in A\) mit \(n \in \mathbb{N}\) heißt die aussagen l. Formel \( \lnot \varphi\) \textbf{Negation} von \(\varphi\), die aussagen l. Formel \(\varphi_1 \wedge \cdots \wedge \varphi_n\) heißt \textbf{Konjunktion} von \(\varphi_1, \cdots, \varphi_n\) und \(\varphi_1 \vee \cdots \vee \varphi_n\) heißt \textbf{Disjunktion} von \(\varphi_1, \cdots, \varphi_n\).
    \end{itemize}

\mysubsection{Definition (konjunktive Normalform)}
\begin{itemize}
    \item [(i)] Ein \textbf{Literal} ist eine variable oder eine negaltion einer Variablen.
    \item [(ii)] Eine \textbf{disjunktive Klausel} ist eine Disjunktion von Literalen.
    \item [(iii)] Eine aussagen l. Formel in \textbf{konjunktiver Normalform}, kurz KNF ist eine Konjunktion disjunkiver klauseln
\end{itemize}

\mysubsection{Definition (Wahrheitswert)}
    Eine \textbf{Belegung} ist eine Funktion \(b : Var \to \{0,1\}\). Der \textbf{Wahrheitswert} \(val_b(\varphi)\) einer Formel \(\varphi\) für eine Belegung b ist induktiv wie folg definiert. 
    \vspace*{0.5cm}
    \\
    \(
        \forall a \in Var \text{ sei } val_b(a) = b(a) \text{ und für } \varphi, \psi \in A \text{ sei} 
    \)
    \begin{equation}
        val_b(\lnot \varphi) := 
        \begin{cases}
            1 & val_b(\varphi) = 0\\
            0 & val_b(\varphi) = 1\\
        \end{cases}    
    \end{equation}
    und 
    \begin{equation}
        val_b(\varphi \wedge \psi) := 
        \begin{cases}
            1 & val_b(\varphi) = val_b(\psi) = 1\\
            0 & \text{sonst}\\
        \end{cases}    
    \end{equation}
    Eine aussagenlogische Formel \(\varphi\) heißt \textbf{erfüllbar}, wenn es eine Belegung b gibt, sodass \(val_b = 1\) gilt.

\mysubsection{Definition (logisch äquivalent)}
    Zwei aussagenlogische Formel \(\psi\) und \(\varphi\) sind \textbf{logisch äquivalent} falls \(val_b(\varphi) = val_b(\psi) \quad \forall\) Belegungen b gilt.

\mysubsection{Definition (SAT)}
    Wir setzen 
    \[
        SAT := \{\varphi \in A : \varphi \text{ist in KNF und erfüllbar}\}  
    \]

\mysubsection{Bemerkung (Das Erfüllbarkeitsproblem SAT in der Komplexitätsklasse NP)*}
    Es gibt SAT \(\in\) \textbf{NP}.

\newpage

\mysubsection{Satz (Satz von Cock)}
    Die Sprache SAT ist \textbf{NP}-vollständig. 
    \begin{proof}
        Nach \hyperref[subsec:8.11]{Bemerkung 8.11} gilt SAT \(\in\) \textbf{NP}. Es genügt also zu zeigen, dass es \(\forall p \in poly\) und p-zeitbeschränkten TM M eine Funktion g \(\in\) \textbf{FP} gibt, sodass \(\forall w \in \{0, 1\}^*\) die aussagenlogische From g(w) ind KNF genau dann erfüllbar ist, wenn \(w \in L(M)\). Dafür wählen wir g(w) so, dass g(w) in wesentlich der Aussage u entspricht. Sei \(c \in \mathbb{N}_0\) und \(p: \mathbb{N} \to \mathbb{N}\), \(n \to n^c + c\) und sei \(M = (Q, \{0, 1\}, \Gamma, \Delta, s, F)\) eine p-zeitbeschränkte TM. \hyperref[subsec:7.10]{Satz 7.10} und \hyperref[subsec:7.4]{Satz 7.4} erlauben uns anzunehmen, dass M eine 1-TM mit Bandalphabet \(\Gamma = \{\Box, 0, 1\}\) ist. Sei \(w \in \{0, 1\}^*\) und \(n :=|w|\).
        \vspace*{0.5cm}
        \\
        Wir setzen \(I:= [p(n)]\) und \(J: \{-p(n)+1, \cdots, p(n)+1\}\). Wir definieren nun eine ganze Ansammlung an variablen, die die Arbeitsweise von M bei eingabe w codieren.
        \begin{center}
            \setlength{\extrarowheight}{3pt} % Adjust the length as needed
            \begin{tabular}{l|p{10cm}}
                Variable & Aussage \\
                \hline
                \hline
                \(T_{i, j, a}\) mit \(i \in I, j \in J, a \in \Gamma\) & Zum Zeitpunkt i ist a das Symbol auf Feld j \\
                \hline
                \(p_{i, j}\) mit \(i \in I, j \in J\) & Zum Zeitpunkt i steht der Kopf auf dem Feld j \\
                \hline
                \(S_{i, q}\) mit \(i \in I, q \in Q\) & Zum Zeitpunkt ist der Zustand q \\
                \hline
                \(D_{i, 0}\) & Der i-te 'Listeneintrag' ist eine Wiederholung einer vorangehenden Stoppkonfiguration. \\
                \hline
                \(D_{i, d}\) mit \(i \in I\) und \(d \in \Delta\) & Die Instruktion d bezeugt im Sinne \hyperref[subsec:2.3]{Definition 2.3(Nachfolgekonfiguration)}, dass der (i+1)-te 'Listeneintrag' Nachfolgekonfiguration des i-ten 'Listeneintrags ist' \\
            \end{tabular}    
        \end{center}
        \paragraph*{\(\leadsto\)}
            Wir haben eine Liste/ Logbuch mit den variablen \(D_{i, 0}\), \(D_{i, d}\), die im i-ten Feld beschreiben was M im i-ten Schritt der Rechnung macht.
        
        Es bleibt g(w) zu konstruieren, so dass \(val_b(g(w)) = 1\) für eine Belegung b gilt, wenn b eine Liste darstellt, die einer rechnung von M zur Eingabe w (gegebenenfalls mit Wiederholungen der Stoppkonfiguration) entspricht.
        \vspace*{0.5cm}
        \\
        Wir geben dazu g(w). Teilformeln die wir nicht als KNF angeben, lassen sich leicht in solche umwandeln. 
        \vspace*{0.5cm}
        \\
        Die Startkonfiguration wird korrekt realisiert:
        \[
            S_{1,S}, \quad P_{1, 1}, \quad \bigwedge \limits_{j\in [n]} T_{1,j, w(j)},\quad \bigwedge \limits_{J/[n]} T_{1, j, \Box}   
        \]
        Zu jedem zeitpunkt sind die relevanten Felder mit höchstens einem Symbol beschriftet:
        \[
            \bigwedge \limits_{(i, j, a, a') \in I \times J \times \Gamma \times \Gamma : a \not = a'} (T_{i, j, a} \to \lnot T_{i, j, a'})
        \]
        Zu jedem Zeitpunkt gibt es höchstens eine aktuelle Kopfposition.
        \[
            \bigwedge \limits_{(i, j, j') \in I \times J \times J : j \not = j'} (P_{i, j} \to \lnot P_{i, j'})      
        \]
        Zu jedem Zeitpunkt gibt es hächstens eine aktuellen Zustand
        \[
            \bigwedge \limits_{(i, q, q') \in I \times Q \times Q : q \not = q'} (S_{i, q} \to \lnot S_{i, q'})      
        \]
        Zu jedem Zeitunkt gibt es höchstens eine auszuführende Instruktion oder es wird die vorherige wieerholt:
        \[
            \bigwedge \limits_{(i, d, d') \in I \times \Delta^+ \times \Delta^+ : d \not = d'} (D_{i, d} \to \lnot D_{i, d'})      
        \]
        Zu jedem Zeitpunkt kann sich nur das Feld ändern auf dem der Kopf steht: 
        \[
            \bigwedge \limits_{(i, j, a) \in I^- \times J \times \Gamma} ((T_{i, j, a \wedge \lnot P_{i, j}}) \to T_{i+1, j, a})      
        \]
        Wird eine Stoppkonfiguration wiederholt, ändert sich das Band, Kopfposition und Zustand nicht:
        \[
            \bigwedge \limits_{(i, j, a) \in I^- \times J \times \Gamma} ((D_{i,0} \wedge T_{i, j, a}) \to T_{i+ 1, j, a} )
        \]
        \[
            \bigwedge \limits_{(i, j) \in I^- \times J } ((D_{i,0} \wedge P_{i, j}) \to P_{i+ 1, j} )
        \]
        \[
            \bigwedge \limits_{(i,q) \in I^- \times Q } ((D_{i,0} \wedge S_{i, q}) \to S_{i+ 1, q})
        \]
        Zu denem Zeitpunkt gibt es eine auszuführende Instruktion oder es wird die vorherige Konfiguration wiederholt:
        \[
            \bigwedge \limits_{i \in I} \bigvee \limits_{d \in \Delta^+} D_{i, d}
        \]
        Konfigurationen, die den Bedingungsteil einer Instruktion erfüllen werden nicht wiederholt:
        \[
            \bigwedge \limits_{(i, j(q, a, a', a', B)) \in I^- \times J \times \Delta} ((S_{i, q} \wedge P_{i, j} \wedge T_{i, j, a}) \to \lnot D_{i, 0})  
        \]
        Gibt es zu einer Konfiguration eine auszuführende Instruktion, so muss deren Bedingungsteil durch die Konfiguration erfüllt werden
        \[
            \bigwedge \limits_{(i, (q, a, q', a', B)) \in I^- \times \Delta} D_{i, (q, a, q',a',B)} \to S_{i,q}
        \]  
        \[
            \bigwedge \limits_{(i, j,(q, a, q', a', B)) \in I^- \times J \times  \Delta} (D_{i, (q, a, q',a',B) \wedge P_{i, j}} \to T_{i,j, a})
        \]  
        Wird eine Instruktion angewendet, so muss die Nachfolgekonfiguration aus dieser Instruktion aus der Vorgängerkonfiguration hervorgehen:
        \[
            \bigwedge \limits_{(i, (q, a, q', a', B)) \in I^- \times \Delta} (D_{i, (q, a, q',a',B)} \to S_{i + 1, q'})
        \]  
        \[
            \bigwedge \limits_{(i, j, (q, a, q', a', B)) \in I^- \times J \times \Delta} (D_{i, (q, a, q',a',B)} \wedge P_{i, j} \to T_{i + 1, j, a})
        \] 
        \[
            \bigwedge \limits_{(i, j, (q, a, q', a', B)) \in I^- \times J \times \Delta} (D_{i, (q, a, q',a',B)}\wedge P_{i, j} \to P_{i + 1, j, + \delta_B})
        \]  
        wobei \(\delta_L := -1\), \(\delta_S := 0\), \(\delta_R:= 1\).
        \vspace*{0.5cm}
        \\
        Es wird eine Stoppkonfiguration erreicht und zum Zeitpunkt p(n) ist der Zustand akzeptierend:
        \[
            D_{p(n), 0}, \quad \bigvee \limits_{q \in F} S_{p(n), q}.    
        \]
        Nach Konstrukton ist damit g(w) genau dann erfüllbar, wenn es eine endliche Rechnung von M zur Eingabe w gibt, die höchstens Länge p(n) hat, also genau, wenn \(w \in L(M)\) gilt.
    \end{proof}

\newpage

\mysubsection{Definition (k-konjunktive Normalform)}
    Für \(k \in \mathbb{N}\) ist eine aussagenlogische Formel in k-konjunktive Normalform, kurz k-KNF, falls sie eine Konjunktion disjunkiver Klauseln, die jeweils die Länge hächstens k haben, ist.

\mysubsection{Definition (k-SAT)}
    Für \(k \in \mathbb{N}\)
    \[
        k-SAT := \{ \varphi \in A : \varphi \text{ist im k-KNF und erfüllbar}\}     
    \]

\mysubsection{Bemerkung(k-SAT als Entscheidungsproblem in der Komplexitätsklasse NP)*}
    Für \(k \in \mathbb{N}\) gilt k-SAT \(\in\) \textbf{NP}.

\mysubsection{Satz (NP-Vollständigkeit von k-SAT für k \(\leq\) 3)*}
    Für \(k \leq 3\) ist k-SAT \textbf{NP}- vollständig.

    \[SKIZZE HIER\]

    \[
        (x \vee y )
    \]
    \[
        \downarrow
    \]

    \[
        (\frac{\quad x \quad }{} \vee a) \wedge (\lnot a \vee \frac{}{\quad y \quad})
    \]
    Nach \hyperref[subsec:8.15]{Bemerkung 8.15} genügt es SAT \( \leq^p_m\) 3-SAT. Wir müssen also eine beliebige aussagenlogische Formel \(\varphi\) in KNF in eine Formel in 3-KNF überführen, sodass die \(\varphi \in\) SAT \(\Leftrightarrow \varphi \in\) 3-SAT****. Sei \(\varphi \in\) SAT und wir konstruieren \(\varphi\)' in 3-SAT wie folgt:
    \vspace*{0.5cm}
    \\
    Wir geben für jede Klausel von \(\varphi\) eine Konjunktion von Klauseln für \(\varphi\) an, die logisch äquivalent ist. Sei \((l,v \cdots vl_r)\) eine Klausel von \(\varphi\). Sei \(r \leq 4\), sonst ist nichts zu tun.
    
    \[
        (l, va_2) \wedge (\bigwedge \limits_{j = 2}^{r = 1} (\lnot a_j \wedge l_j \wedge a_{j+1})) \wedge (\lnot a_r \wedge l_r)  \quad \circledast 
    \]
    Hier sind \(a_2, \cdots, a_r\) neue Variablen. 
    \vspace*{0.5cm}
    \\
    Wir fügen \(\circledast\) für \((l, v \cdots vl_r)\) zu \(\varphi'\) hinzu und verfahren für jede andere Klausel analog. Nun kann man überprüfem, dass \(\varphi\) und \(\varphi '\)

    \[SKIZZE HIER\]

\mysubsection{Definition (Graphen (V, E))\(\diamondsuit\)}
    Ein \textbf{Graph} ist ein Paar (V,E), wobei V eine endliche Menge ist, die \textbf{Echenmenge}, und \(E \subseteq 2^V\) eine Menge zweielementiger Mengen, die \textbf{Kantenmenge}, ist.

\mysubsection{Definition (k-Färbung)}
    Für \(k \in \mathbb{N}\) ist eine k-Färbung eines graphen (V,E) eine Funktion \(\phi : V \to [k]\), sodass \(\phi(u) \not = \phi(v) \quad \forall \{u, v\} \in E\) gilt.

\mysubsection{Definition (k-COLORING)}
    Es sei 
    \[
        \text{COLORING} := \{(G,k) : \text{es gibt eine k-färbung des Graphen G}\}
    \]
    und für \(k \in \mathbb{N}\) sei
    \[
        \text{k-COLORING} := \{G : \text{es gibt eine k-Färbung für den Graphen G}\}  
    \]

\mysubsection{Bemerkung(NP-Eigenschaft COLORING und k-COLORING)\(\diamondsuit\)}
    Es gibt COLORING \(\in\) \textbf{NP} und für \(k \in \mathbb{N}\) gilt k-COLORING \(\in\) \textbf{NP}.

\mysubsection{Satz(NP-Vollständig 3-Coloring)\(\diamondsuit\)}
    Das Problem 3-Coloring ist \textbf{NP}-vollständig.
    \begin{proof}
        Nach \hyperref[subsec:8.20]{Bemerkung 8.20} und \hyperref[subsec:8.16]{Satz 8.16} genügt es 3-SAT \(\leq^P_m\) 3-COLORING zu zeigen. Wir müssen also eine aussagenlogische Formel \(\varphi\) in einen Graphen G transformieren \(\cdots\)

        \(\varphi\) in 3-KNF
    \end{proof}

\mysubsection{Korrolar (k-COLORING NP-vollständig )\(\diamondsuit\)}
    Für \(k \leq 3\) ist k-COLORING \textbf{NP}-vollständig.
    \begin{proof}
        Übung
    \end{proof}

\mysubsection{Korrolar(COLORING NP-vollständig)\(\diamondsuit\)}
    COLORING ist \textbf{NP}-vollständig.

\mysubsection{Definition (EXACTCOVER)}
    Für eine Menge S von Mengen ist eine exakte Überdeckung von S eine Teilmenge T \(T \subseteq S\) paarweise disjunkive Menge mit \(\bigcup \limits_{A\in T} A = \bigcup \limits_{A \in S} A\). 
    \[
        \text{EXACTCVER} := \{S : \text{S ist eine endliche Menge endlicher Mengen mit einer exakten Überdeckung}\}  
    \]
\mysubsection{Bemerkung (EXACTCOVER NP Eigenschaft)\(\diamondsuit\)}
    EXACTCOVER \(\in\) \textbf{NP}

\mysubsection{Satz (EXACTCOVER NP-vollständig)\(\diamondsuit\)}
    EXACTCOVER ist \textbf{NP}-vollständig.
    \[\text{ohne beweis.}\]

\mysubsection{Definition (SUBSETSUM)}
    Sei
    \[
        \text{SUBSETSUM} := \{(a_1, \cdots, a_n, b) \in \mathbb{N}_0^{n+1} : \exists x_1, \cdots x_n \in \{0,1\} : \sum_{i}^{} x_i a_i = b\}    
    \]

\mysubsection{Bemerkung(SUBSETSUM NP Eigenschaft)*}
    SUBSETSUM \(\in\) \textbf{NP}

\mysubsection{Satz(SUBSETSUM NP-vollständig)\(\diamondsuit\)}
    SUBSETSUM ist \textbf{NP}-vollständig.
    \[\text{ohne beweis.}\]